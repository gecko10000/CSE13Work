\documentclass[12pt]{article}
\usepackage{fullpage}

\title{Design Document - Numerical Integration}
\author{Lev Teytelman}
\date{\today}

\begin{document}\maketitle
\section{Program Description}

This program calculates and prints integrals of functions over an interval using the Composite Simpson's 1/3 Rule. The functions themselves use various converging series to get an estimate of the outcomes.
\section{Included Files}

\begin{itemize}
    \item functions.c - The source file providing functions corresponding to each input flag.
    \item functions.h - The header file providing the function declarations for use in other files.
    \item integrate.c - The file containing the \verb|main()| function, argument processing, and usage help.
    \item mathlib.c - The source file with the implementations for various mathematical functions using converging series.
    \item mathlib.h - The header file with the declarations of the mathematical functions.
    \item test.c - Test cases for mathlib.c, testing function versatility.
    \item Makefile - The logic for compiling and linking files into one executable binary.
    \item README.md - A short explanation of the program and instructions for usage.
    \item DESIGN.pdf - This document, describes the program, included files, and how it works.
    \item WRITEUP.pdf - The document explaining the code in-depth and analyzing the results.
\end{itemize}
\section{Structure}

This assignment includes implementing converging series to calculate specific values such as square root, log, sin, cos, and others. The basic structure of these is presented in Python, and the goal is to turn it into a modular C function to calculate the expected outcome. These functions are then passed into and used by an \verb|integrate| function, which approximates the integral of a function over an interval by partitioning the area into small rectangles. All this finally comes together in the user-facing program, where the user must specify the function to be calculated using program flags, the low and high values for the interval to calculate, and optionally the number of partitions.
\section{Pseudocode}

Python pseudocode is provided for many of the functions; however, for \verb|Cos| and \verb|integrate|, it is not. The code for the \verb|Cos| function is similar to \verb|Sin|, except \verb|v| and \verb|t| are $1$ instead of $x$, representing $x^0$ instead of $x^1$, and $k$ needs to be $2$ to have an even factorial divisor.

The \verb|integrate| function calculates the area under the curve using the Composite Simpson's 1/3 Rule and will be structured similarly to the following pseudocode:
\begin{verbatim}
    x(j) is a function to calculate x via a + jh
    function is a pointer passed into the integrate function
    
    total = 0
    total += function(x(0))
    for i in range(1, n/2 - 1):
        total += 2 * function(x(2i))
    for i in range(1, n/2):
        total += 4 * function(x(2i-1))
    total += function(x(n))
    return total * h / 3
\end{verbatim}
\end{document}


